\documentclass[spanish,a4paper]{article} 

\usepackage{babel}
\usepackage{soul}
\usepackage[latin1]{inputenc}
%\usepackage{bbm}
\usepackage{framed}
\makeatletter

\@ifclassloaded{beamer}{%
  \newcommand{\tocarEspacios}{%
    \addtolength{\leftskip}{4em}%
    \addtolength{\parindent}{-3em}%
  }%
}
{%
  \usepackage[top=1cm,bottom=2cm,left=1cm,right=1cm]{geometry}%
  \usepackage{color}%
  \newcommand{\tocarEspacios}{%
    \addtolength{\leftskip}{5em}%
    \addtolength{\parindent}{-3em}%
  }%
}

\newcommand{\encabezadoDeProblema}[4]{%
  % Ponemos la palabrita problema en tt
%  \noindent%
  {\normalfont\bfseries\ttfamily problema}%
  % Ponemos el nombre del problema
  \ %
  {\normalfont\ttfamily #2}%
  \ 
  % Ponemos los parametros
  (#3)%
  \ifthenelse{\equal{#4}{}}{}{%
  \ =\ %
  % Ponemos el nombre del resultado
  {\normalfont\ttfamily #1}%
  % Por ultimo, va el tipo del resultado
  \ : #4}
}

\newcommand{\encabezadoDeTipo}[2]{%
  % Ponemos la palabrita tipo en tt
  {\normalfont\bfseries\ttfamily tipo}%
  % Ponemos el nombre del tipo
  \ %
  {\normalfont\ttfamily #2}%
  \ifthenelse{\equal{#1}{}}{}{$\langle$#1$\rangle$}
}

% Primero definiciones de cosas al estilo title, author, date

\def\materia#1{\gdef\@materia{#1}}
\def\@materia{No especifi\'o la materia}
\def\lamateria{\@materia}

\def\cuatrimestre#1{\gdef\@cuatrimestre{#1}}
\def\@cuatrimestre{No especifi\'o el cuatrimestre}
\def\elcuatrimestre{\@cuatrimestre}

\def\anio#1{\gdef\@anio{#1}}
\def\@anio{No especifi\'o el anio}
\def\elanio{\@anio}

\def\fecha#1{\gdef\@fecha{#1}}
\def\@fecha{\today}
\def\lafecha{\@fecha}

\def\nombre#1{\gdef\@nombre{#1}}
\def\@nombre{No especific'o el nombre}
\def\elnombre{\@nombre}

\def\practicas#1{\gdef\@practica{#1}}
\def\@practica{No especifi\'o el n\'umero de pr\'actica}
\def\lapractica{\@practica}


% Esta macro convierte el numero de cuatrimestre a palabras
\newcommand{\cuatrimestreLindo}{
  \ifthenelse{\equal{\elcuatrimestre}{1}}
  {Primer cuatrimestre}
  {\ifthenelse{\equal{\elcuatrimestre}{2}}
  {Segundo cuatrimestre}
  {Verano}}
}


\newcommand{\depto}{{UBA -- Facultad de Ciencias Exactas y Naturales --
      Departamento de Computaci\'on}}

\newcommand{\titulopractica}{
  \centerline{\depto}
  \vspace{1ex}
  \centerline{{\Large\lamateria}}
  \vspace{0.5ex}
  \centerline{\cuatrimestreLindo de \elanio}
  \vspace{2ex}
  \centerline{{\huge Pr\'actica \lapractica -- \elnombre}}
  \vspace{5ex}
  \arreglarincisos
  \newcounter{ejercicio}
  \newenvironment{ejercicio}{\stepcounter{ejercicio}\textbf{Ejercicio
      \theejercicio}%
    \renewcommand\@currentlabel{\theejercicio}%
  }{\vspace{0.2cm}}
}  


\newcommand{\titulotp}{
  \centerline{\depto}
  \vspace{1ex}
  \centerline{{\Large\lamateria}}
  \vspace{0.5ex}
  \centerline{\cuatrimestreLindo de \elanio}
  \vspace{0.5ex}
  \centerline{\lafecha}
  \vspace{2ex}
  \centerline{{\huge\elnombre}}
  \vspace{5ex}
}


%practicas
\newcommand{\practica}[2]{%
    \title{Pr\'actica #1 \\ #2}
    \author{Algoritmos y Estructuras de Datos I}
    \date{Primer Cuatrimestre 2015}

    \maketitle
}


% Simbolos varios

\newcommand{\ent}{\ensuremath{\mathbb{Z}}}
\newcommand{\float}{\ensuremath{\mathbb{R}}}
\newcommand{\bool}{\ensuremath{\mathsf{Bool}}}
\newcommand{\True}{\ensuremath{\mathrm{True}}}
\newcommand{\False}{\ensuremath{\mathrm{False}}}
\newcommand{\Then}{\ensuremath{\rightarrow}}
\newcommand{\Iff}{\ensuremath{\leftrightarrow}}
\newcommand{\implica}{\ensuremath{\longrightarrow}}
\newcommand{\IfThenElse}[3]{\ensuremath{\mathsf{if}\ #1\ \mathsf{then}\ #2\ \mathsf{else}\ #3}}


\newcommand{\rango}[2]{[#1\twodots#2]}
\newcommand{\comp}[2]{[\,#1\,|\,#2\,]}

\newcommand{\rangoac}[2]{(#1\twodots#2]}
\newcommand{\rangoca}[2]{[#1\twodots#2)}
\newcommand{\rangoaa}[2]{(#1\twodots#2)}

%ejercicios
\newtheorem{exercise}{Ejercicio}
\newenvironment{ejercicio}{\begin{exercise}\rm}{\end{exercise} \vspace{0.2cm}}
\newenvironment{items}{\begin{enumerate}[i)]}{\end{enumerate}}
\newenvironment{subitems}{\begin{enumerate}[a)]}{\end{enumerate}}
\newcommand{\sugerencia}[1]{\noindent \textbf{Sugerencia:} #1}

%tipos basicos
\newcommand{\rea}{\ensuremath{\mathsf{Float}}}
\newcommand{\cha}{\ensuremath{\mathsf{Char}}}

\newcommand{\mcd}{\mathrm{mcd}}
\newcommand{\prm}[1]{\ensuremath{\mathsf{prm}(#1)}}
\newcommand{\sgd}[1]{\ensuremath{\mathsf{sgd}(#1)}}
\newcommand{\trd}[1]{\ensuremath{\mathsf{trd}(#1)}} %agregado por Olga Isakova

%listas
\newcommand{\TLista}[1]{[#1]}
\newcommand{\lvacia}{\ensuremath{[\ ]}}
\newcommand{\lv}{\ensuremath{[\ ]}}
\newcommand{\longitud}[1]{\left| #1 \right|}
\newcommand{\cons}[1]{\ensuremath{\mathsf{cons}}(#1)}
\newcommand{\indice}[1]{\ensuremath{\mathsf{indice}}(#1)}
\newcommand{\conc}[1]{\ensuremath{\mathsf{conc}}(#1)}
\newcommand{\cab}[1]{\ensuremath{\mathsf{cab}}(#1)}
\newcommand{\cola}[1]{\ensuremath{\mathsf{cola}}(#1)}
\newcommand{\sub}[1]{\ensuremath{\mathsf{sub}}(#1)}
\newcommand{\en}[1]{\ensuremath{\mathsf{en}}(#1)}
\newcommand{\cuenta}[2]{\mathsf{cuenta}\ensuremath{(#1, #2)}}
\newcommand{\suma}[1]{\mathsf{suma}(#1)}
\newcommand{\twodots}{\ensuremath{\mathrm{..}}}
\newcommand{\masmas}{\ensuremath{++}}

% Acumulador
\newcommand{\acum}[1]{\ensuremath{\mathsf{acum}}(#1)}
\newcommand{\acumselec}[3]{\ensuremath{\mathrm{acum}(#1 |  #2, #3)}}

% \selector{variable}{dominio}
\newcommand{\selector}[2]{#1~\ensuremath{\leftarrow}~#2}
\newcommand{\selec}{\ensuremath{\leftarrow}}


\newenvironment{problema}[4][res]{%
  % El parametro 1 (opcional) es el nombre del resultado
  % El parametro 2 es el nombre del problema
  % El parametro 3 son los parametros
  % El parametro 4 es el tipo del resultado
  % Preambulo del ambiente problema
  % Tenemos que definir los comandos requiere, asegura, modifica y aux
  \newcommand{\requiere}[2][]{%
    {\normalfont\bfseries\ttfamily requiere}%
    \ifthenelse{\equal{##1}{}}{}{\ {\normalfont\ttfamily ##1} :}\ %
    \ensuremath{##2}%
    {\normalfont\bfseries\,;\par}%
  }
  \newcommand{\asegura}[2][]{%
    {\normalfont\bfseries\ttfamily asegura}%
    \ifthenelse{\equal{##1}{}}{}{\ {\normalfont\ttfamily ##1} :}\
    \ensuremath{##2}%
    {\normalfont\bfseries\,;\par}%
  }
  \newcommand{\modifica}[1]{%
    {\normalfont\bfseries\ttfamily modifica\ }%
    \ensuremath{##1}%
    {\normalfont\bfseries\,;\par}%
  }
  \renewcommand{\aux}[4]{%
    {\normalfont\bfseries\ttfamily aux\ }%
    {\normalfont\ttfamily ##1}%
    \ifthenelse{\equal{##2}{}}{}{\ (##2)}\ : ##3\, = \ensuremath{##4}%
    {\normalfont\bfseries\,;\par}%
  }
  \newcommand{\res}{#1}
  \vspace{1ex}
  \noindent
  \encabezadoDeProblema{#1}{#2}{#3}{#4}
  % Abrimos la llave
  \{\par%
  \tocarEspacios
}
% Ahora viene el cierre del ambiente problema
{
  % Cerramos la llave
  \noindent\}
  \vspace{1ex}
}


 \newcommand{\aux}[4]{%
    {\normalfont\bfseries\ttfamily aux\ }%
    {\normalfont\ttfamily #1}%
    \ifthenelse{\equal{#2}{}}{}{\ (#2)}\ : #3\, = \ensuremath{#4}%
    {\normalfont\bfseries\,;\par}%
  }


\newcommand{\pre}[1]{\textsf{pre}\ensuremath{(#1)}}

\newcommand{\problemanom}[1]{\textsf{#1}}
\newcommand{\problemail}[3]{\textsf{problema #1}\ensuremath{(#2) = #3}}
\newcommand{\problemailsinres}[2]{\textsf{problema #1}\ensuremath{(#2)}}
\newcommand{\requiereil}[2]{\textsf{requiere #1: }\ensuremath{#2}}
\newcommand{\asegurail}[2]{\textsf{asegura #1: }\ensuremath{#2}}
\newcommand{\modificail}[1]{\textsf{modifica }\ensuremath{#1}}
\newcommand{\auxil}[2]{\textsf{aux }\ensuremath{#1 = #2}}
\newcommand{\auxilc}[4]{\textsf{aux }\ensuremath{#1( #2 ): #3 = #4}}
\newcommand{\auxnom}[1]{\textsf{aux }\ensuremath{#1}}

\newcommand{\comentario}[1]{{/*\ #1\ */}}

\newcommand{\nom}[1]{\ensuremath{\mathsf{#1}}}

% -----------------
% Tipos compuestos
% -----------------

\newcommand{\Pred}[1]{\mathit{#1}}
\newcommand{\TSet}[1]{\textsf{Conjunto}\ensuremath{\langle #1 \rangle}}
\newcommand{\TSetFinito}[1]{\textsf{Conjunto}\ensuremath{\langle #1 \rangle}}
\newcommand{\TRac}{\tiponom{Racional}}
\newcommand{\TVec}{\tiponom{Vector}}
\newcommand{\Func}[1]{\mathrm{#1}}
\newcommand{\cardinal}[1]{\left| #1 \right|}


\newcommand{\sinonimo}[2]{%
  \noindent%
  {\normalfont\bfseries\ttfamily tipo\ }%
  #1\ =\ #2%
  {\normalfont\bfseries\,;\par}
}

\newcommand{\enum}[2]{%
  \noindent%
  {\normalfont\bfseries\ttfamily tipo\ }%
  #1\ =\ #2%
  {\normalfont\bfseries\,;\par}
}

%~ \newenvironment{tipo}[1]{%
    %~ \vspace{0.2cm}
    %~ \textsf{tipo #1}\ensuremath{\{}\\
    %~ \begin{tabular}[l]{p{0.02\textwidth} p{0.02\textwidth} p{0.82 \textwidth}}
%~ }{%
    %~ \end{tabular}
%~ 
    %~ \ensuremath{\}}
    %~ \vspace{0.15cm}
%~ }
%~ 

\newenvironment{tipo}[2][]{%
  % Preambulo del ambiente tipo
  % Tenemos que definir los comandos observador (con requiere) y aux
  \newcommand{\observador}[3]{%
    {\normalfont\bfseries\ttfamily observador\ }%
    {\normalfont\ttfamily ##1}%
    \ifthenelse{\equal{##2}{}}{}{\ (##2)}\ : ##3%
    {\normalfont\bfseries\,;\par}%
  }
  \newcommand{\requiere}[2][]{{%
    \addtolength{\leftskip}{3em}%
    \setlength{\parindent}{-2em}%
    {\normalfont\bfseries\ttfamily requiere}%
    \ifthenelse{\equal{##1}{}}{}{\ {\normalfont\ttfamily ##1} :}\ 
    \ensuremath{##2}%
    {\normalfont\bfseries\,;\par}}
  }
  \newcommand{\explicacion}[1]{{%
    \addtolength{\leftskip}{3em}%
    \setlength{\parindent}{-2em}%
    \par \hspace{2.3em} ##1 %
    {\par}
    }
  }
  \newcommand{\invariante}[2][]{%
    {\normalfont\bfseries\ttfamily invariante}%
    \ifthenelse{\equal{##1}{}}{}{\ {\normalfont\ttfamily ##1} :}\ 
    \ensuremath{##2}%
    {\normalfont\bfseries\,;\par}%
  }
  \renewcommand{\aux}[4]{%
    {\normalfont\bfseries\ttfamily aux\ }%
    {\normalfont\ttfamily ##1}%
    \ifthenelse{\equal{##2}{}}{}{\ (##2)}\ : ##3\, = \ensuremath{##4}%
    {\normalfont\bfseries\,;\par}%
  }
  \vspace{1ex}
  \noindent
  \encabezadoDeTipo{#1}{#2}
  % Abrimos la llave
  \{\par%
  \tocarEspacios
}
% Ahora viene el cierre del ambiente tipo
{
  % Cerramos la llave
  \noindent\}
  \vspace{1ex}
}


%~ \newcommand{\observador}[3]{%
    %~ & \multicolumn{2}{p{0.85\textwidth}}{\textsf{observador #1}\ensuremath{(#2):#3}}\\%
    %~ }
    
%~ \newcommand{\observador}[3]{%
    %~ {\normalfont\bfseries\ttfamily observador\ }%
    %~ {\normalfont\ttfamily ##1}%
    %~ \ifthenelse{\equal{##2}{}}{}{\ (##2)}\ : ##3%
    %~ {\normalfont\bfseries\,;\par}%
%~ }
    

%~ \newcommand{\observadorconreq}[3]{
    %~ & \multicolumn{2}{p{0.85\textwidth}}{\textsf{observador #1}\ensuremath{(#2):#3 \{}}\\
%~ }
%~ \newcommand{\observadorconreqfin}{
    %~ & \multicolumn{2}{p{0.85\textwidth}}{\ensuremath{\}}}\\
%~ }
%~ \newcommand{\obsrequiere}[2][]{& & \textsf{requiere #1: }\ensuremath{#2};\\}
%~ 
%~ \newcommand{\explicacion}[1]{&& #1 \\}
%~ \newcommand{\invariante}[2][]{%
    %~ & \multicolumn{2}{p{0.85\textwidth}}{\textsf{invariante #1: }\ensuremath{#2}}\\%
%~ }
%~ \newcommand{\auxinvariante}[2]{
    %~ & \multicolumn{2}{p{0.85\textwidth}}{\textsf{aux }\ensuremath{#1 = #2}};\\
%~ }
%~ \newcommand{\auxiliar}[4]{
    %~ & \multicolumn{2}{p{0.85\textwidth}}{\textsf{aux }\ensuremath{#1(#2): #3 = #4}};\\
%~ }

\newcommand{\tiponom}[1]{\ensuremath{\mathsf{#1}}\xspace}
\newcommand{\obsnom}[1]{\ensuremath{\mathsf{#1}}}

% -----------------
% Ecuaciones de terminacion en funcional
% -----------------

\newenvironment{ecuaciones}{%
    $$
    \begin{array}{l @{\ /\ (} l @{,\ } l @{)\ =\ } l}
}{%
    \end{array}
    $$
}




\newcommand{\ecuacion}[4]{#1 & #2 & #3 & #4\\}

\newcommand{\concat}{\nom{concat}}

% Listas por comprension. El primer parametro es la expresion y el
% segundo tiene los selectores y las condiciones.
%*\newcommand{\comp}[2]{[\,#1\,|\,#2\,]}























% En las practicas/parciales usamos numeros arabigos para los ejercicios.
% Aca cambiamos los enumerate comunes para que usen letras y numeros
% romanos
\newcommand{\arreglarincisos}{%
  \renewcommand{\theenumi}{\alph{enumi}}
  \renewcommand{\theenumii}{\roman{enumii}}
  \renewcommand{\labelenumi}{\theenumi)}
  \renewcommand{\labelenumii}{\theenumii)}
}





%%%%%%%%%%%%%%%%%%%%%%%%%%%%%% PARCIAL %%%%%%%%%%%%%%%%%%%%%%%%
\let\@xa\expandafter
\newcommand{\tituloparcial}{\centerline{\depto -- \lamateria}
  \centerline{\elnombre -- \lafecha}%
  \setlength{\TPHorizModule}{10mm} % Fija las unidades de textpos
  \setlength{\TPVertModule}{\TPHorizModule} % Fija las unidades de
                                % textpos
  \arreglarincisos
  \newcounter{total}% Este contador va a guardar cuantos incisos hay
                    % en el parcial. Si un ejercicio no tiene incisos,
                    % cuenta como un inciso.
  \newcounter{contgrilla} % Para hacer ciclos
  \newcounter{columnainicial} % Se van a usar para los cline cuando un
  \newcounter{columnafinal}   % ejercicio tenga incisos.
  \newcommand{\primerafila}{}
  \newcommand{\segundafila}{}
  \newcommand{\rayitas}{} % Esto va a guardar los \cline de los
                          % ejercicios con incisos, asi queda mas bonito
  \newcommand{\anchodegrilla}{20} % Es para textpos
  \newcommand{\izquierda}{7} % Estos dos le dicen a textpos donde colocar
  \newcommand{\abajo}{2}     % la grilla
  \newcommand{\anchodecasilla}{0.4cm}
  \setcounter{columnainicial}{1}
  \setcounter{total}{0}
  \newcounter{ejercicio}
  \setcounter{ejercicio}{0}
  \renewenvironment{ejercicio}[1]
  {%
    \stepcounter{ejercicio}\textbf{Ejercicio \theejercicio. [##1
      puntos]}% Formato
    \renewcommand\@currentlabel{\theejercicio}% Esto es para las
                                % referencias
    \renewcommand{\invariante}[2]{%
      {\normalfont\bfseries\ttfamily invariante}%
      \ ####1\hspace{1em}####2%
    }%
    \renewcommand{\problema}[5][result]{
      \encabezadoDeProblema{####1}{####2}{####3}{####4}\hspace{1em}####5}%
  }% Aca se termina el principio del ejercicio
  {% Ahora viene el final
    % Esto suma la cantidad de incisos o 1 si no hubo ninguno
    \ifthenelse{\equal{\value{enumi}}{0}}
    {\addtocounter{total}{1}}
    {\addtocounter{total}{\value{enumi}}}
    \ifthenelse{\equal{\value{ejercicio}}{1}}{}
    {
      \g@addto@macro\primerafila{&} % Si no estoy en el primer ej.
      \g@addto@macro\segundafila{&}
    }
    \ifthenelse{\equal{\value{enumi}}{0}}
    {% No tiene incisos
      \g@addto@macro\primerafila{\multicolumn{1}{|c|}}
      \bgroup% avoid overwriting somebody else's value of \tmp@a
      \protected@edef\tmp@a{\theejercicio}% expand as far as we can
      \@xa\g@addto@macro\@xa\primerafila\@xa{\tmp@a}%
      \egroup% restore old value of \tmp@a, effect of \g@addto.. is
      
      \stepcounter{columnainicial}
    }
    {% Tiene incisos
      % Primero ponemos el encabezado
      \g@addto@macro\primerafila{\multicolumn}% Ahora el numero de items
      \bgroup% avoid overwriting somebody else's value of \tmp@a
      \protected@edef\tmp@a{\arabic{enumi}}% expand as far as we can
      \@xa\g@addto@macro\@xa\primerafila\@xa{\tmp@a}%
      \egroup% restore old value of \tmp@a, effect of \g@addto.. is
      % global 
      % Ahora el formato
      \g@addto@macro\primerafila{{|c|}}%
      % Ahora el numero de ejercicio
      \bgroup% avoid overwriting somebody else's value of \tmp@a
      \protected@edef\tmp@a{\theejercicio}% expand as far as we can
      \@xa\g@addto@macro\@xa\primerafila\@xa{\tmp@a}%
      \egroup% restore old value of \tmp@a, effect of \g@addto.. is
      % global 
      % Ahora armamos la segunda fila
      \g@addto@macro\segundafila{\multicolumn{1}{|c|}{a}}%
      \setcounter{contgrilla}{1}
      \whiledo{\value{contgrilla}<\value{enumi}}
      {%
        \stepcounter{contgrilla}
        \g@addto@macro\segundafila{&\multicolumn{1}{|c|}}
        \bgroup% avoid overwriting somebody else's value of \tmp@a
        \protected@edef\tmp@a{\alph{contgrilla}}% expand as far as we can
        \@xa\g@addto@macro\@xa\segundafila\@xa{\tmp@a}%
        \egroup% restore old value of \tmp@a, effect of \g@addto.. is
        % global 
      }
      % Ahora armo las rayitas
      \setcounter{columnafinal}{\value{columnainicial}}
      \addtocounter{columnafinal}{-1}
      \addtocounter{columnafinal}{\value{enumi}}
      \bgroup% avoid overwriting somebody else's value of \tmp@a
      \protected@edef\tmp@a{\noexpand\cline{%
          \thecolumnainicial-\thecolumnafinal}}%
      \@xa\g@addto@macro\@xa\rayitas\@xa{\tmp@a}%
      \egroup% restore old value of \tmp@a, effect of \g@addto.. is
      \setcounter{columnainicial}{\value{columnafinal}}
      \stepcounter{columnainicial}
    }
    \setcounter{enumi}{0}%
    \vspace{0.2cm}%
  }%
  \newcommand{\tercerafila}{}
  \newcommand{\armartercerafila}{
    \setcounter{contgrilla}{1}
    \whiledo{\value{contgrilla}<\value{total}}
    {\stepcounter{contgrilla}\g@addto@macro\tercerafila{&}}
  }
  \newcommand{\grilla}{%
    \g@addto@macro\primerafila{&\textbf{TOTAL}}
    \g@addto@macro\segundafila{&}
    \g@addto@macro\tercerafila{&}
    \armartercerafila
    \ifthenelse{\equal{\value{total}}{\value{ejercicio}}}
    {% No hubo incisos
      \begin{textblock}{\anchodegrilla}(\izquierda,\abajo)
        \begin{tabular}{|*{\value{total}}{p{\anchodecasilla}|}c|}
          \hline
          \primerafila\\
          \hline
          \tercerafila\\
          \tercerafila\\
          \hline
        \end{tabular}
      \end{textblock}
    }
    {% Hubo incisos
      \begin{textblock}{\anchodegrilla}(\izquierda,\abajo)
        \begin{tabular}{|*{\value{total}}{p{\anchodecasilla}|}c|}
          \hline
          \primerafila\\
          \rayitas
          \segundafila\\
          \hline
          \tercerafila\\
          \tercerafila\\
          \hline
        \end{tabular}
      \end{textblock}
    }
  }%
  \vspace{0.4cm}
  \textbf{LU:}
  
  \textbf{Apellidos:}
  
  \textbf{Nombres:}
  \vspace{0.5cm}
}



% AMBIENTE CONSIGNAS
% Se usa en el TP para ir agregando las cosas que tienen que resolver
% los alumnos.
% Dentro del ambiente hay que usar \item para cada consigna

\newcounter{consigna}
\setcounter{consigna}{0}

\newenvironment{consignas}{%
  \newcommand{\consigna}{\stepcounter{consigna}\textbf{\theconsigna.}}%
  \renewcommand{\ejercicio}[1]{\item ##1 }
  \renewcommand{\problema}[5][result]{\item
    \encabezadoDeProblema{##1}{##2}{##3}{##4}\hspace{1em}##5}%
  \newcommand{\invariante}[2]{\item%
    {\normalfont\bfseries\ttfamily invariante}%
    \ ##1\hspace{1em}##2%
  }
  \renewcommand{\aux}[4]{\item%
    {\normalfont\bfseries\ttfamily aux\ }%
    {\normalfont\ttfamily ##1}%
    \ifthenelse{\equal{##2}{}}{}{\ (##2)}\ : ##3 \hspace{1em}##4%
  }
  % Comienza la lista de consignas
  \begin{list}{\consigna}{%
      \setlength{\itemsep}{0.5em}%
      \setlength{\parsep}{0cm}%
    }
}%
{\end{list}}



% para decidir si usar && o ^
\newcommand{\y}[0]{\ensuremath{\land}}

% macros de correctitud
\newcommand{\semanticComment}[2]{#1 \ensuremath{#2};}
\newcommand{\namedSemanticComment}[3]{#1 #2: \ensuremath{#3};}


\newcommand{\local}[1]{\semanticComment{local}{#1}}

\newcommand{\vale}[1]{\semanticComment{vale}{#1}}
\newcommand{\valeN}[2]{\namedSemanticComment{vale}{#1}{#2}}
\newcommand{\impl}[1]{\semanticComment{implica}{#1}}
\newcommand{\implN}[2]{\namedSemanticComment{implica}{#1}{#2}}
\newcommand{\estado}[1]{\semanticComment{estado}{#1}}

\newcommand{\invarianteCN}[2]{\namedSemanticComment{invariante}{#1}{#2}}
\newcommand{\invarianteC}[1]{\semanticComment{invariante}{#1}}
\newcommand{\varianteCN}[2]{\namedSemanticComment{variante}{#1}{#2}}
\newcommand{\varianteC}[1]{\semanticComment{variante}{#1}}

\usepackage{ifthen}
\usepackage{amssymb}
\usepackage{multicol}
\usepackage[absolute]{textpos}

\begin{document}

\materia{Algoritmos y Estructura de Datos I}
\cuatrimestre{1}
\anio{2015}

\fecha{1 de Abril de 2015}

\nombre{\LARGE TPE - Flores vs Vampiros}

\titulotp

\section{Tipos}

\enum{Habilidad}{Generar, Atacar, Explotar}
\enum{ClaseVampiro}{Caminante, Desviado}
\sinonimo{Posicion}{(\ent, \ent)}
\sinonimo{Vida}{\ent}



\section{Flor}
\begin{tipo}{Flor}{}
	\observador{vida}{f: Flor}{\ent}
	\observador{cuantoPega}{f: Flor}{\ent}
	\observador{habilidades}{f: Flor}{[Habilidad]}
	\medskip
	\invariante{sinRepetidos(habilidades(f))}
	\invariante[lasHabilidadesDeterminanLaVidayElGolpe]
	{\\ vida(f) == 100 \ \textbf{div} \ (\longitud{habilidades(f)}+1) \land \\
	cuantoPega(f) == \IfThenElse{Atacar \in habilidades(f)}{12 \ \textbf{div} \
	\longitud{habilidades(f)}}{0} }
\end{tipo}


\begin{problema}{nuevaF}{v : \ent, cP : \ent, hs : [Habilidad]}{Flor}
\requiere{v == 100 \ \textbf{div} \ (\longitud{hs} + 1)}
\requiere{cP == \IfThenElse{Atacar \in hs}{12 \ \textbf{div} \ \longitud{hs}}{0}}
\requiere{\longitud{hs} \geq 1}
\requiere{sinRepetidos(hs)}
\asegura{vida(\res) == v }
\asegura{cuantoPega(\res) == cP}
\asegura{mismos(habilidades(\res), hs)}
\end{problema}

\begin{problema}{vidaF}{f: Flor}{\ent}
\asegura{\res == vida(f)}
\end{problema}

\begin{problema}{cuantoPegaF}{f: Flor}{\ent}
\asegura{\res == cuantoPega(f)}
\end{problema}

\begin{problema}{habilidadesF}{f: Flor}{[Habilidad]}
\asegura{mismos(\res, habilidades(f))}
\end{problema}



\section{Vampiro}
\begin{tipo}{Vampiro}{}
	\observador{clase}{v: Vampiro}{ClaseVampiro}
	\observador{vida}{v: Vampiro}{\ent}
	\observador{cuantoPega}{v: Vampiro}{\ent}
	\medskip
	\invariante[vidaEnRango]{vida(v) \geq 0 \land vida(v) \leq 100}
	\invariante[pegaEnSerio]{cuantoPega(v) > 0}
\end{tipo}

\begin{problema}{nuevoV}{cV : ClaseVampiro, v : \ent, cP : \ent}{Vampiro}
\requiere{v \geq 0 \land v \leq 100}
\requiere{cP > 0}
\asegura{clase(res) == cV}
\asegura{vida(res) == v}
\asegura{cuantoPega(res) == cP}
\end{problema}

\begin{problema}{claseVampiroV}{v : Vampiro}{ClaseVampiro}
\asegura{\res == clase(v)}
\end{problema}

\begin{problema}{vidaV}{v : Vampiro}{\ent}
\asegura{\res == vida(v)}
\end{problema}

\begin{problema}{cuantoPegaV}{v : Vampiro}{\ent}
\asegura{\res == cuantoPega(v)}
\end{problema}

% problemas igual a Flor

\section{Nivel}
\begin{tipo}{Nivel}{}
	\observador{ancho}{n: Nivel}{\ent}
	\observador{alto}{n: Nivel}{\ent}
	\observador{turno}{n: Nivel}{\ent}
	\observador{soles}{n: Nivel}{\ent}
	\observador{flores}{n: Nivel}{[(Flor, Posicion, Vida)]}
	\observador{vampiros}{n: Nivel}{[(Vampiro, Posicion, Vida)]}
	\observador{spawning}{n: Nivel}{[(Vampiro, \ent, \ent)]}
	\invariante[valoresRazonables]{ancho(n) > 0 \land alto(n) > 0 \land soles(n) \geq 0 \land turno(n) \geq 0}
	
	\invariante[posicionesValidas]{((\forall
	\selector{f}{flores(n)})\ 0 < fila(f) \leq alto(n)
	\land 0 < columna(f) \leq ancho(n)) \land ((\forall
	\selector{v}{vampiros(n)})\ 0 < fila(v) \leq alto(n)
	\land 0 \leq columna(v) \leq ancho(n))}
	
	\invariante[spawningOrdenado]{(\forall
	\selector{i}{[0..|spawning(n)|-1))}\ (\trd{spawning(n)_{i}} <
	\trd{spawning(n)_{i+1}}) \lor (\trd{spawning(n)_{i}} == \trd{spawning(n)_{i+1}} \land \sgd{spawning(n)_{i}} \leq
	\sgd{spawning(n)_{i+1}})}
	
	\invariante[necesitoMiEspacio]{(\forall i,j \selec [0..|flores(n)|), i \neq j) sgd(flores(n)_i) \neq sgd(flores(n)_j)}
	\invariante[vivosPeroNoTanto]{vidaFloresOk(flores(n)) \land vidaVampirosOk(vampiros(n))}
	\invariante[spawneanBien]{(\forall t \selec spawning(n))sgd(t)\geq 1 \land sgd(t) \leq alto(n) \land trd(t)\geq 0}
\end{tipo}


\begin{problema}{nuevoN}{an : \ent, al : \ent, s : \ent, spaw : [(Vampiro, \ent, \ent)]}{Nivel}
\requiere{\longitud{spaw} > 0}
\requiere{an > 0}
\requiere{al > 0}
\requiere{s \geq 0}
\requiere{(\forall t \leftarrow spaw)\ \sgd{t} \geq 1 \land \sgd{t} \leq al \land \trd{t} \geq 0}
\requiere{(\forall \selector{i}{[0..|spaw|-1))}(\trd{spaw_{i}} <
\trd{spaw_{i+1}}) \lor (\trd{spaw_{i}} == \trd{spaw_{i+1}}
\land \\ \sgd{spaw_{i}} \leq \sgd{spaw_{i+1}})}
\asegura{ancho(\res) == an}
\asegura{alto(\res) == al}
\asegura{turno(\res) == 0}
\asegura{spawning(\res) == spaw}
\asegura{flores(\res) == []}
\asegura{vampiros(\res) == []}
\end{problema}

\begin{problema}{anchoN}{n : Nivel}{\ent}
\asegura{\res == ancho(n)}
\end{problema}

\begin{problema}{altoN}{n : Nivel}{\ent}
\asegura{\res == alto(n)}
\end{problema}


\begin{problema}{turnoN}{n : Nivel}{\ent}
\asegura{\res == turno(n)}
\end{problema}

\begin{problema}{solesN}{n : Nivel}{\ent}
\asegura{\res == soles(n)}
\end{problema}

\begin{problema}{floresN}{n : Nivel}{[(Flor, Posicion, Vida)]}
\asegura{mismasFloresDeNivel(n, \res)}
\end{problema}

\begin{problema}{vampirosN}{n : Nivel}{[(Vampiro, Posicion, Vida)]}
\asegura{mismos(\res, vampiros(n))}
\end{problema}

\begin{problema}{spawningN}{n : Nivel}{[(Vampiro, \ent, \ent)]}
\asegura{\res == spawning(n)}
\end{problema}

\begin{problema}{comprarSoles}{n: Nivel, s : \ent}{}
\requiere{s>0}
\modifica{n}
\asegura{soles(n)==soles(\pre{n})+s}
\asegura{ancho(n)==ancho(\pre{n}) \land alto(n)==alto(\pre{n}) \land
turno(n)==turno(\pre{n})}
\asegura{mismasFloresDeNivel(n, flores(\pre{n}))}
\asegura{mismos(vampiros(\pre{n}),vampiros(n))}
\asegura{spawning(\pre{n}) == spawning(n)} % Cambie los mismos porque el orden debe mantenerse
\end{problema}

\begin{problema}{obsesivoCompusilvo}{n: Nivel}{\bool}
\asegura{(\forall \selector{x, y}{flores(n)}, x \neq y)
sucesor(x, y, n)\implica{atacante(x) \neq
atacante(y)}}
\aux{atacante}{f: (Flor, Posicion, Vida)}{\bool}{Atacar \in
habilidades(\prm{f})}

% si x e y son elementos de flores(n) distintos, el invariante
% necesitoMiEspacio garantiza que Posicion(x)!=Posicion(y), luego en flores(n)
% no puede haber ni elementos con la misma Posicion, ni elementos repetidos
\aux{sucColumna}{x, s: (Flor, Posicion, Vida), niv:
Nivel}{\bool}{\\fila(x)==fila(s) \land columna(x)<columna(s) \land \\ 
\neg (\exists \selector{i}{flores(niv)}, i \neq x, i \neq s) (fila(x)==fila(i)
\land columna(i) \in \rangoaa{columna(x)}{columna(s)})}

\aux{sucesor}{x, s: (Flor, Posicion, Vida), niv:
Nivel}{\bool}{\\sucColumna(x, s, niv) \lor (fila(x)<fila(s) \land \neg
hayIntermedio(x, s, niv)}

\aux{hayIntermedio}{x, s: (Flor, Posicion,
Vida), niv: Nivel}{\bool}{\\(\exists \selector{i}{flores(niv)}, i \neq x, i \neq
s) sucColumna(x, i) \lor sucColumna(i, s) \lor fila(i) \in
\rangoaa{fila(x)}{fila(s)}}

% el patr�n alternante se observa si para cada elemento de flores, su sucesor
% inmediato tiene valor de 'atacante' distinto (True o False).
\end{problema}

\begin{problema}{agregarFlor}{n: Nivel, f : Flor, p : Posicion}{}
% verifica que la posici�n es v�lida y no est� ocupada 
\requiere[posicionOk]{0 < \prm{p} \leq alto(n) \land 0 < \sgd{p} \leq ancho(n) \land
\neg (\exists \selector{x}{flores(n)}) \sgd{x} == \sgd{f}}
% verifica que hay soles suficientes para la flor
\requiere[solesOk]{soles(n) \geq solesReq(f)}
\modifica{n}
\asegura{ancho(n)==ancho(\pre{n}) \land alto(n)==alto(\pre{n}) \land
turno(n)==turno(\pre{n})}
\asegura{mismos(vampiros(\pre{n}), vampiros(n))}
\asegura{spawning(n) == spawning(\pre{n})}
\asegura{mismasFloresDeNivel(n, (f, p, vida(f)):flores(\pre{n}))}
\asegura{soles(n)==soles(\pre{n})-solesReq(f)}
\aux{solesReq}{f: Flor}{\ent}{2^{\longitud{habilidades(f)}}}
\end{problema}

\noindent\aux{terminado}{n: Nivel}{\bool}{\longitud{vampiros(n)}==0 \lor
(\exists \selector{v}{vampiros(n)}) columna(v)==0}

\begin{problema}{pasarTurno}{n: Nivel}{}
\requiere{\neg{terminado(n)}}
\modifica{n}
\asegura{ancho(n)==ancho(\pre{n}) \land alto(n)==alto(\pre{n})} 
\asegura{turno(n) == turno(\pre{n})+1}
\asegura{soles(n) == (soles(\pre{n})+solesRecaudados(\pre{n}) + 1)}

% al pasar el turno quedan las flores sobrevivientes..
\asegura{mismasFloresDeNivel(n, \comp{(\prm{f}, \sgd{f}, \trd{f}-ataqueFlor(f,
\pre{n}))}{\selector{f}{flores(\pre{n})}, \\ sobrevivienteFlor(f, \pre{n})})}
% ..y quedan los vampiros sobrevivientes
\asegura{mismos(vampiros(n), avanzan(\pre{n}) \masmas quedan(\pre{n}) \masmas
retroceden(\pre{n}) \masmas spawnean(\pre{n}))}

% la flor es sobreviviente si al recibir ataque sigue con vida y no explota
\aux{sobrevivienteFlor}{f: (Flor, Posicion, Vida), niv: Nivel}{\bool}
{\trd{f}-ataqueFlor(f, niv) > 0 \land \\ \neg (\exists
\selector{v}{vampiros(niv)}) explosion(f, v)}

% la flor explota si tiene habilidad de explotar y hay alg�n vampiro 
% en su posici�n
\aux{explosion}{f: (Flor, Posicion, Vida), v: (Vampiro, Posicion. Vida)}{\bool}
{\\ Explotar \in habilidades(\prm{f}) \land \sgd{f}==\sgd{v}}

% ataqueFlor es la cantidad de HP que pierde la flor al pasar el turno -
% por cada vampiro con quien comparte la posici�n, es lo que pega ese vampiro
\aux{ataqueFlor}{f:(Flor,Posicion,Vida), niv:Nivel}{\ent}
{\\ \suma{[cuantoPega(\prm{v})|\selector{v}{vampiros(niv)}, \sgd{f}==\sgd{v}]}}

\aux{sobrevivienteVamp}{v: (Vampiro, Posicion, Vida), niv:
Nivel}{\bool}{\trd{v}-ataqueVamp(v, \pre{n})>0}

% avanzan los vampiros sobrevivientes que no tienen una flor en su posicion 
\aux{avanzan}{niv: Nivel}{[(Vampiro, Posicion,
Vida)]}{\\ \comp{(\prm{v}, (nuevaFila(v), columna(v)-1), \trd{v}-ataqueVamp(v,
niv)}{\selector{v}{vampiros(niv)}, \\ sobrevivienteVamp(v, niv),
\neg (\exists \selector{f}{flores(niv)}) fila(f)==fila(v) \land
columna(f)==columna(v)}}

% quedan en el mismo lugar los que comparten la posici�n con una flor que no
% explota
\aux{quedan}{niv: Nivel}{[(Vampiro, Posicion,
Vida)]}{\\ \comp{(\prm{v}, (nuevaFila(v), columna(v)), \trd{v}-ataqueVamp(v,
niv)}{\selector{v}{vampiros(niv)}, \\ sobrevivienteVamp(v, niv), (\exists
\selector{f}{flores(niv)}) \neg explosion(f, v) \land fila(f)==fila(v) \land \\
columna(f)==columna(v)}}

% retroceden los que comparten la posici�n con una flor que explota
\aux{retroceden}{niv: Nivel}{[(Vampiro, Posicion,
Vida)]}{\\ \comp{(\prm{v}, (nuevaFila(v), columna(v)+1), \trd{v}-ataqueVamp(v,
niv)}{\selector{v}{vampiros(niv)}, \\ sobrevivienteVamp(v, niv), (\exists
\selector{f}{flores(niv)}) explosion(f, v) \land fila(f)==fila(v) \land \\
columna(f)==columna(v)}}

% los que tienen el turno que viene en la lista d\foralle spawning aparecen en el
% extremo derecho
\aux{spawnean}{niv: Nivel}{[(Vampiro, Posicion,
Vida)]}{\\ \comp{(\prm{v}, (\sgd{v}, ancho(niv)),
vida(\prm{v}))}{\selector{v}{spawning(niv)}, \trd{v}==turno(niv)+1}}

% la fila cambia si el vampiro es desviado
\aux{nuevaFila}{v: (Vampiro, Posicion,
Vida)}{\ent}{\\ \IfThenElse{clase(\prm{v})==Desviado \land fila(v)
\neq 1}{\sgd{v}-1}{\sgd{v}}}
\end{problema}

\begin{problema}{estaEnJaque}{n: Nivel}{Vampiro}
\asegura{\res \in vampirosEnJaque(n)}
\aux{vampirosEnJaque}{niv: Nivel}{[Vampiro]}{\comp{\prm{v}}{v \selec vampiros(niv), \\
(\forall w \selec vampiros(niv)) ataqueVamp(v, niv) \geq ataqueVamp(w, niv)}}
\end{problema}


\section{Juego}
\begin{tipo}{Juego}
	\observador{flores}{j: Juego}{[Flor]}
	\observador{vampiros}{j: Juego}{[Vampiro]}
	\observador{niveles}{j: Juego}{[Nivel]}
	\invariante[floresDistintas]{(\forall i, k \selec [0..\longitud{flores(j)}), i \neq k) \neg floresIguales(flores(j)_i, flores(j)_k)}
	\invariante[vampirosDistintos]{sinRepetidos(vampiros(j))}
	\invariante[nivelesConFloresValidas]{\\ (\forall
	\selector{n}{niveles(j)})(\forall \selector{f}{flores(n)})(\exists \selector{x}{flores(j)})
	floresIguales(\prm{f}, x)} 
	\invariante[nivelesConVampirosValidos]{(\forall
	\selector{n}{niveles(j)})(\forall \selector{v}{vampiros(n)})\prm{v} \in
	vampiros(j)}
\end{tipo}

\begin{problema}{floresJ}{j: Juego}{[Flor]}
\asegura{mismasListasDeFlores(\res, flores(j))}
\end{problema}

\begin{problema}{vampirosJ}{j: Juego}{[Vampiro]}
\asegura{mismos(\res, vampiros(j))}
\end{problema}

\begin{problema}{nivelesJ}{j: Juego}{[Nivel]}
\asegura {mismosNivelesDeJuego(j, \res)}
% Importa el orden de los niveles en el juego, 
% no se puede usar mismasListasDeNiveles
\aux{mismosNivelesDeJuego}{jg: Juego, ns: [Nivel]}{\bool}
{\\ \longitud{niveles(j)}==\longitud{ns} \land 
(\forall k \selec \rangoca{0}{\longitud{niveles(jg)}})
nivelesIguales(niveles(jg)_{k}, ns_{k})}
\end{problema}

\begin{problema}{agregarNivelJ}{j: Juego, n: Nivel, i: \ent}{}
\requiere{0 \leq i \leq |niveles(j)|}
\requiere{turno(n) == 0}
\requiere{|flores(n)| == 0}
\requiere{|vampiros(n)| == 0}
\modifica{j}
\asegura{nivelesIguales(niveles(j)_{i}, n)}
\asegura{mismasListasDeFlores(flores(j), flores(pre(j)))}
\asegura{mismos(vampiros(j), vampiros(pre(j)))}
% Al agregar un nivel en el indice i, la lista de niveles antes de i queda
% igual, mientras despu�s de i est� corrida un �ndice a la derecha
\asegura{((\forall \selector{x}{[0..i)}) nivelesIguales(niveles(j)_{x},
niveles(pre(j))_{x}) \land \\ ((\forall
\selector{y}{(i..\longitud{niveles(j)})}) nivelesIguales(niveles(j)_{y},
niveles(pre(j))_{y-1})}
\end{problema}

\begin{problema}{estosSalenFacil}{j: Juego}{[Nivel]}
\asegura{mismasListasDeNiveles(masPlantas(masSoles(niveles(j))), \res)}
\aux{masSoles}{ns: [Nivel]}{[Nivel]}{\comp{n}{\selector{n}{ns},
tieneMaxSoles(n, ns)}}
\aux{masPlantas}{ns: [Nivel]}{[Nivel]}{\comp{n}{\selector{n}{ns},
tieneMaxPlantas(n, ns)}}
\aux{tieneMaxSoles}{n: Nivel, ns: [Nivel]}{Bool}{(\forall niv \selec ns)
soles(niv) \leq soles(n)} 
\aux{tieneMaxPlantas}{n: Nivel, ns: [Nivel]}{Bool}{(\forall niv \selec ns)
\longitud{flores(niv)} \leq \longitud{flores(n)}}
\end{problema}

\begin{problema}{jugarNivel}{j: Juego, n: Nivel, i: \ent}{}
% Estado futuro: Requiere mismo alto y ancho, los soles son dinamicos por lo que pueden variar
% El turno actual debe ser mayor o igual
% La lista de spawning va a tener el mismo sufijo, pero si los turnos avanzaron la misma tuvo que
% haberse limpiado progresivamente.
\requiere{0 \leq i < long(niveles(j))}
\requiere{esMismoAltoYAncho(niveles(j)_{i}, n)}
\requiere{spawningFuturo(niveles(j)_{i}, n)}
\requiere{turnoFuturo: turno(niveles(j)_{i}) \leq turno(n))}
\requiere{nivelConFloresValidas: (\forall \selector{f}{flores(n)})(\exists \selector{x}{flores(j)})floresIguales(\prm{f}, x)}
\requiere{nivelConVampirosValidos: (\forall \selector{v}{vampiros(n)})\prm{v} \in vampiros(j)}
\modifica{j}
\asegura{esMismoAltoYAncho(niveles(j)_{i}, n)}
\asegura{mismos(vampiros(niveles(j)_{i}), vampiros(n))}
\asegura{mismasFloresDeNivel(niveles(j)_{i}, flores(n))}
\asegura{spawning(niveles(j)_{i}) == spawning(n)}
\asegura{soles(niveles(j)_{i}) == soles(n)}
\asegura{turno(niveles(j)_{i}) == turno(n)}
\asegura{mismasListasDeFlores(flores(j), flores(\pre{j}))}
\asegura{mismos(vampiros(j), vampiros(\pre{j})}
\asegura{\longitud{niveles(j)}==\longitud{niveles(\pre{j})} \land \\ (\forall
\selector{k}{\rangoca{0}{\longitud{niveles(j)}}}, k \neq i) nivelesIguales(niveles(j)_k, niveles(\pre{j}_k))}

% spawningFuturo detecta si el spawning de nf es el sufijo del spawning de ni,
% o sea un spawning futuro posible  
\aux{spawningFuturo}{ni, nf : Nivel}{\bool}
{\\ (\exists k \selec \rangoca{0}{\longitud{spawning(ni)}}) 
(\forall m \selec \rangoca{k}{\longitud{spawning(ni)}}) ni_m==nf_{m-k}}
\aux{esMismoAltoYAncho}{ni, nf : Nivel}{Bool}{(alto(ni)==alto(nf))\land(ancho(ni)==ancho(nf))}
\end{problema}

\begin{problema}{altoCheat}{j: Juego, i: \ent}{}
\requiere{0 \leq i < |niveles(j)|}
\modifica{j}
\asegura{mismasListasDeFlores(flores(j), flores(\pre{j}))}
\asegura{mismos(vampiros(j), vampiros(\pre{j}))}
\asegura{mismos(listaVampirosNivel(j, i), reducirVidas(listaVampirosNivel(pre(j), i)))}
\asegura{ancho(niveles(j)_i)==ancho(niveles(\pre{j})_i)) \land
alto(niveles(j)_i)==alto(niveles(\pre{j})_i)) \land \\ 
soles(niveles(j)_i)==soles(niveles(\pre{j})_i)) \land
spawning(niveles(j)_i)==spawning(niveles(\pre{j})_i)) \land \\
mismasFloresDeNivel(niveles(j)_{i}, flores(niveles(\pre{j})_{i}))} 
\asegura{(\forall \selector{k}{[0..\longitud{niveles(j)}), k \neq i]} )
nivelesIguales(niveles(j)_k, niveles(\pre{j})_k)} 

\aux{listaVampirosNivel}{jg: Juego, i: \ent}{[(Vampiro, Posicion, Vida)]}{vampiros(niveles(jg)_{i})}

\aux{reducirVidas}{vs: [(Vampiro, Posicion, Vida)]}{[(Vampiro, Posicion, Vida)]}
{\\ \comp{(\prm{v}, \sgd{v}, \trd{v} \ \textbf{div} \ 2)}{\selector{v}{vs}}}
\end{problema}

\begin{problema}{muyDeExactas}{j: Juego}{\bool}
\asegura{res == esFibo(soloNivelesGanados(j))}
\aux{soloNivelesGanados}{j : Juego}{[\ent]}{\comp{i+1}{i \selec
[0..\longitud{niveles(j)}), esGanado(niveles(j)_i)}}
\aux{esGanado}{n: Nivel}{Bool}{terminado(n) \land long(vampiros(n)) == 0 }
\aux{esFibo}{s : [\ent]}{Bool}{s_0==1 \land s_1==2 \land (\forall i \selec
[1..\longitud{s}-2])(s_{\longitud{s}-i} == s_{\longitud{s}-i-1} +
s_{\longitud{s}-i-2})}
\end{problema}

\begin{problema}{nivelesSoleados}{j: Juego}{[Nivel]}
\asegura{mismasListasDeNiveles(\res ,listaNivelesNoTerminados(j))}
\asegura{ordenadoPorSoles(\res)}
\aux{listaNivelesNoTerminados}{j: Juego}{[Nivel]}{\comp{n}{n \leftarrow niveles(j),�terminado(n)}}
\aux{ordenadoPorSoles}{xs: [Nivel]}{\bool}{((\forall i \leftarrow [0..|xs|-1)) \ (soles(xs_{i}) + solesRecaudados(xs_{i}) + 1) \geq (soles(xs_{i+1}) + solesRecaudados(xs_{i+1}) + 1)) \ \lor \ ((\forall j \leftarrow [0..|xs|-1)) \ (soles(xs_{j}) + solesRecaudados(xs_{j}) + 1) \leq (soles(xs_{j+1}) + solesRecaudados(xs_{j+1}) + 1))}
\end{problema}

\section{Auxiliares}
\aux{vidaFloresOk}{fs: [(Flor, Posicion, Vida)]}{\bool}{(\forall f \selec fs) trd(f) > 0 \land trd(f) \leq vida(prm(f))}
\aux{vidaVampirosOk}{fs: [(Vampiro, Posicion, Vida)]}{\bool}{(\forall f \selec fs) trd(f) > 0 \land trd(f) \leq vida(prm(f))}
\aux{fila}{x: (T, Posicion, Vida)}{\ent}{\prm{\sgd{x}}}
\aux{columna}{x: (T, Posicion, Vida)}{\ent}{\sgd{\sgd{x}}}
\aux{floresIguales}{x, y: Flor}{\bool}{mismos(habilidades(x), habilidades(y))}
\aux{mismos}{a, b: \TLista{T}}{\bool}{\longitud{a}==\longitud{b} \land
(\forall x \in a) \cuenta{x}{a} == \cuenta{x}{b}}
\aux{mismasFloresDeNivel}{n: Nivel, fs: [(Flor, Posicion, Vida)]}{\bool}
{\\ \indent \indent \longitud{flores(n)}==\longitud{fs} \land 
((\forall \selector{f}{flores(n)}) cuentaFloresDeNivel(f, flores(n))== cuentaFloresDeNivel(x, fs))}
\aux{mismasListasDeFlores}{xs, ys:[Flor]}{\bool}{\\ \indent
\indent \longitud{xs}==\longitud{ys} \land (\forall
\selector{x}{xs})cuentaFlores(x, xs) == cuentaFlores(y, ys)}
\aux{mismasListasDeNiveles}{xs,ys : [Nivel]}{\bool}
{\\ \indent \indent |xs|==|ys| \land
(\forall x \leftarrow xs)cuentaNiveles(x, xs) == cuentaNiveles(y, ys)} \aux{nivelesIguales}{n, k: Nivel}{\bool}{ancho(n)==ancho(k) \land
alto(n)==alto(k) \land soles(n)==soles(k) \land \\ \indent \indent
mismos(vampiros(n), vampiros(k)) \land spawning(n)==spawning(k) \land
mismasFloresDeNivel(n, flores(k)) \land \\ \indent \indent turno(n)==turno(k)}
\aux{cuentaNiveles}{n: Nivel, ns: [Nivel]}{\ent}{|[m | \selector{m}{ns}, nivelesIguales(n, m)]|}
\aux{cuentaFlores}{f: Flor, fs: [Flor]}{\ent}{|[g | \selector{g}{fs}, floresIguales(f, g)]|}
\aux{cuentaFloresDeNivel}{f: (Flor, Posicion, Vida), fs: [(Flor, Posicion, Vida)]}{\ent}
{\\ \indent \indent |[g | \selector{g}{fs}, floresIguales(\prm{g}, \prm{f})
\land \sgd{g}==\sgd{f} \land \trd{g} == \trd{f}]|}

%ataqueVamp es la cantidad de HP que pierde el vampiro al pasar el turno - la
% suma de lo que pega cada flor en la fila del vampiro, en la columna menor o
% igual a la del vampiro, sin vampiros en el medio
\aux{ataqueVamp}{v: (Vampiro, Posicion, Vida), niv: Nivel}{\ent}
{\\ \indent \indent \suma{[cuantoPega(\prm{f})|\selector{f}{flores(niv)},
fila(f)==fila(v), columna(f) \leq columna(v), \\ \indent \indent \neg (\exists
\selector{w}{vampiros(niv)}) fila(v)==fila(w) \land columna(w)<columna(v)]}}

\aux{solesRecaudados}{niv:Nivel}{\ent}{\suma{[1|\selector{f}{flores(niv)},
Generar \in habilidades(\prm{f})]}}

\end{document}
